\documentclass{article} % For LaTeX2e
\usepackage{nips14submit_e,times}
\usepackage{hyperref}
\usepackage{url}
\usepackage{graphicx}
\usepackage{amsmath}
\usepackage{amssymb}
\usepackage{anyfontsize}
\usepackage{subcaption}
\usepackage{titling}
\usepackage{array}
\usepackage{authblk}

\title{Latent Predicate Networks: unsupervised learning of  probabilistic programs with context sensitive grammars}

\author{Eyal Dechter\thanks{{\tt \{edechter, rule, jbt\} @mit.edu}}}
\author{Joshua Rule}
\author{Joshua B. Tenenbaum}
\affil{{\normalsize Department of Brain and Cognitive Sciences, MIT}}
\date{}

\newcommand{\fix}{\marginpar{FIX}}
\newcommand{\new}{\marginpar{NEW}}

\nipsfinalcopy % Uncomment for camera-ready version

\begin{document}

\maketitle

\begin{abstract}

  Probabilistic programs successfully capture many aspects of the
  complex symbolic and statistical knowledge brought to bear in human
  cognition. Children aren't born with this knowledge, however, but
  must instead learn it, often in an unsupervised fashion. Compelling
  probabilistic programming models of human cognition must therefore
  also be learned. We present early work on Latent Predicate Networks
  (LPN), a framework for unsupervised learning in a restricted space
  of context-sensitive probabilistic programs. Specifically, we use
  Online Variational Bayesian Expectation Maximization in
  PRISM, a probabilistic programming system, to perform inference over
  the set of Stochastic Logic Programs (SLP) parseable in polynomial
  time. Using a small fragment of English sentences about number, we
  show that LPNs can recover the relevant conceptual structure
  necessary to generate novel sentences in this domain with high
  probability.
  
\end{abstract}

\section{Introduction}

%% Introduction idea
%% *** Probabilistic programs have been great models of human cognition
%% **** one-shot learning
%% **** linguistic pragmatics
%% **** some other example
%% *** one thing they lack is a good way to automatically induce full programs from data
%% *** yet, children seem to do this all the time
%% *** So, this motivates our approach
%% *** But, we have to overcome some technical challenges
%% **** many prob.prog. languages are too expressive to be tractable
%% **** many models of concept learning use too restricted a set of primitives

Probabilistic programs can successfully model many aspects of human
cognition [CITIATONS]. Part of their strength as compared to other
models of cognition lies in their ability to flexibly integrate both
symbolic and statistical information within a single framework.

One area probabilistic programming systems lag behind human cognition
is in their ability to automatically induce entire 

Although concept learning and language acquisition have typically been
treated as distinct problems in AI, linguistics and cognitive
development, they are strongly coupled for a child learning to
understand language. More generally, people learn many abstract
concepts primarily through language even though understanding language
depends on understanding the underlying concepts. Research in concept
learning is often focused on concepts grounded in perceptual features,
and while it is almost certainly true that many concepts are learned
via generalization from concrete examples, some concepts cannot be
learned this way.

Number concepts are a good example: children do not learn about the
meaning of ``seventy five'' by seeing examples of seventy five things;
they do not know that ``seventy five'' is more than ``twenty five''
because of their perceptual experiences of these quantities. Rather,
children learn the meaning of ``seventy five'' (or ``a billion and
five'') by noticing how number words are used in language, in counting
sequences, in arithmetic exercises, {\it etc}. Other good examples of such
abstract concepts are kinship and social relations ({\it e.g.} ``my father
in law's grandmother''), temporal relations (``the day after last
Thanksgiving.''), and spatial relations (``above and just to the left
of'').

Such concepts share many of the properties of language syntax: they
are unbounded in number, they derive their meanings via composition,
and, although people only ever say, hear, or read about a small number
of them, they are able to reason correctly about any them. There seems
to be a grammar to these concepts, and grasping this grammar is
critical to understanding their meanings and how to use them. This
motivates our approach here, which is to apply the tools of
probabilistic grammars more familiar from studies of syntax to the
problem of concept aquisition.

{\bf include somewhere:} A general probabilistic logic
programming system based on PROLOG, PRISM provides built-in predicates
for probabilistic execution and Bayesian inference over logic programs
with stochastic choices.

Doing so requires overcoming some technical barriers.

First, many formal systems of probabilistic programming are so
expressive as to make learning generally intractable. To address this,
we use Range Concatenation Grammar, a context-sensitive grammar
formalism originally developed in linguistics and extend this to a
probabilistic model of programs.

Second, the categories of syntax -- the nonterminals of the grammar --
are often assumed to be known to children innately and given to
automated learners by human experts. The categories that underlie
conceptual knowledge, on the other hand, are far more numerous, vary
from domain to domain, and are unlikely to be known to the
learner. This motivates our use of latent predicates that, through
learning, assume the role of a domain's underlying concepts (in the
number domain, these might correspond to the concepts of
successorship, order of magnitude, magnitude comparison, exact
vs. approximate, {\it etc}.).

Finally, inducing probabilistic context-sensitive grammars with latent
predicates threatens to be intractable: our goal is to find a middle
ground between expressivity and tractability. Using
PRISM~\cite{DBLP:journals/jair/SatoK01} -- a probabilistic logic
programming system that naturally implements efficient dynamic
programming algorithms for our models -- we are able to explore which
domains and which grammar architectures are a good fit for this
grammar-based approach.

The rest of this paper describes early work on this approach. First,
we present Latent Predicate Networks (LPNs), our probabilistic model
of concept learning. Then, we describe our approach to inference and
its implementation in PRISM. Finally, we present preliminary
experimental results in the domain of number concept learning that
demonstrate the promise of this approach.

\section{Latent Predicate Networks}

\begin{figure}[t]
  \begin{subfigure}[b]{0.5\linewidth}
    \includegraphics[width=\linewidth]{lpn/lpn.pdf}
    \caption{The architecture and rules of a schematic LPN.}
    \label{fig:architecture}
  \end{subfigure}
  \hfill
  \begin{subfigure}[b]{0.5\linewidth}
    \includegraphics[width=\linewidth]{parseTree/parse.pdf}
    \caption{A possible parse of the sentence ``after twenty five comes twenty six" using a 5-predicate LPN.}
    \label{fig:parseexample}
  \end{subfigure}
\end{figure}

An LPN is a hierarchical Bayesian model of strings
extending the Hierarchical Dirichlet PCFG model (HD-PCFG) to
Probabilistic Range Concatenation Grammars (PRCGs). 

\subsection{Probabilistic Range Concatenation Grammars}
Range Concatenation Grammars (RCGs) are a class of string grammars
that represent all and only those languages that can be parsed in time
polynomial in the length of the target
string~\cite{boullier2005range}. An RCG $G=(N, T, V, P, S)$ is a
5-tuple where $N$ is a finite set of predicate symbols, $T$ is a set
of terminal symbols, $V$ is a set of variable symbols, P is a finite
set of $M \geq 0$ clauses of the form $\psi_0 \rightarrow \psi_1 \dots
\psi_M$, and $S \in N$ is the \emph{axiom}. Each $\psi_m$ is a term of
the form $A(\alpha_1, \dots, \alpha_{\mathcal{A}(A)}$, where $A \in
N$, $\mathcal{A}(A)$ is the arity of $A$, and each $\alpha_i \in (T
\cup V)^*$ is an argument of $\psi_m$. We call the left hand side term
of any clause the \emph{head} of that clause and its predicate symbol
is the \emph{head predicate}.

A string $x$ is in the language defined by an RCG if one can
\emph{derive} $S(x)$. A derivation is a sequence of rewrite steps in
which substrings of the left hand side argument string are bound to
the variables of the head of some clause, thus determining the
arguments in the clause body. If a clause has no body terms, then its
head is derived; otherwise, its head is derived if its body clauses
are derived.\footnote{This description of the language of an RCG
  technically only holds for \emph{non-combinatory} RCGs, in which the
  arguments of body terms can only contain single variables. Since any
  \emph{combinatory} RCG can be converted into a non-combinatory RCG
  and we only consider non-combinatory RCGs here, this description
  suffices.}

We extend RCGs to PRCGs by annotating each clause $C_k \in P$ with
probabilities $p_k$ such that for all predicates ${A \in N, \,
  \sum_{k:head(C_k)=A} p_k = 1}$. A PRCG defines a distribution over
strings $x$ by sampling from derivations of $S(x)$ according to the
product of probabilities of clauses used in that derivation. This is a
well defined distribution as long as no probability mass is placed on
derivations of infinite length; in this paper, we only consider PRCGs
with derivations of finite length, so we need not worry about this
requirement. See Figure \ref{fig:grammar} for an example of the context-sensitive 2-copy language $\{ww\,|w \in \{a,b\}^+\}$ as a PRCG.

\subsection{Learning Model}


\begin{align*}
  \vec{w}_{A_k} &\sim Dir(\vec{\alpha}_{A_k})\\
  x_j &\underset{iid}{\sim} p_{\text{\textsc{prcg}}}(S(x_j)\,|\,\{w_{A_k}\})\\
  p(\{\vec{w}_{A_k}\}\,|\, \vec{x}, \{\vec{\alpha}_{A_k}\}) &\propto
  \prod_j p_{\text{\textsc{prcg}}}(x_j|\{\vec{w}_{A_k}\}) \prod_{A_k}
  p_{\text{\textsc{dir}}}(\vec{w}_{A_k}\,|\,\vec{\alpha}_{A_k})
\end{align*}

Given a collection of predicates, $\{A_k\}_{k=1}^{K}$, and a
distribution over clauses, $\{\vec{w}_{A_k}\}$, the learning task is
to model a set of outputs, $\{x_j\}_{j=1}^{J}$, as being generated
according to the above distribution. In words, the weights of clauses
sharing head predicate $A_k$ are drawn from a Dirichlet distribution
defined by $\vec{\alpha}_{A_k}$. Each data point $x_j$ is then drawn
from the resulting PRCG. We use Bayes' Rule to update  weights $\{\vec{w}_{A_k}\}$ given our data $\vec{x}$.

\section{Inference \label{sec:implementation}}

Bayesian inference over stochastic grammars and stochastic logic
programs has been an active area of research in recent
decades~\cite{DBLP:journals/etai/Muggleton00, cussens2001parameter,
  DBLP:conf/emnlp/LiangPJK07, goldwater2006contextual,
  johnson2006adaptor}.  Variational inference is a popular approach in
this domain and the one we adopt here. We implemented inference by
translating LPNs into PRISM programs and using its built-in
Variational Bayes Expectation-Maximization
algorithm~\cite{sato2008variational}, modified for for use in an
online, rather than batch, setting. We use the adaptive learning rate
of \cite{ranganath2013adaptive}.

LPNs can be encoded as a restricted subclass of PRISM programs; this
is very similar to how PCFGs are encoded in
PRISM~\cite{DBLP:conf/cl/2000}.  There are several restrictions placed
on PRISM programs to maintain the validity of their probabilistic
interpretation. Most importantly, derivations of a probabilistic
predicate cannot contain cycles. Because we disallow
$A_i(\epsilon,\epsilon)$ as a valid clause, every term in the body of
a clause has shorter arguments than the head, giving acyclic and
finite derivations.

\section{Experiment}

To evaluate LPNs as a probabilistic model of concept acquisition, we
trained an LPN with $4$ latent predicates on a set of sentences
expressing successor and predecessor relations in numbers between one
and ninety-nine. The training set was the collection of sentences $$X
= \{[\text{after}\, | \, \text{before}] \, \langle n \rangle \,
\text{comes} \, \langle n+1 \rangle \,|\, n \in 1,\dots,99\},$$ where
$\langle n \rangle$ is the number word corresponding to $n$. The
lexicon was the set of number word corresponding to $1$ through $19$,
the decades $20, \dots, 30$, the empty string, and the words
``before'' and ``after.'' It is not difficult to manually work out an
LPN that describes this limited domain of sentences; see
Figure~\ref{fig:parseexample}, for a possible LPN derivation of an
example sentence.

Although it is difficult to know how common these kinds of sentences
are in child-directed speech, words for small numbers are far more
common than words for larger ones \cite{macwhinney2000childes}. On the
other hand, children learning to count to large numbers rehearse the
sequence. To approximate this distribution of evidence, we drew these
sample sentences from a sum of a geometric distribution with parameter
$0.5$ and a uniform distribution. These components were weighted
$75\%$ and $25\%$, respectively. We drew $2000$ examples from this
distribution, holding out the sentences in Table~\ref{tab:results} for
evaluation.

For inference, we used default $\frac{1}{D}$ pseudocounts (where $D$ is the
dimensionality of the Dirichlet distributions). We found that different random
initialization for this experiment did not lead to qualitatively different
results, though further investigation will be necessary to see how
robust the algorithm is to local maxima when fitting LPNs.

We evaluated the learned model by asking for Viterbi ({\it i.e.} maximum a
posteriori) completions of the last words of each held out test
sentence. Table~\ref{tab:results} shows these completions. The grammar
correctly learns much of the structure of these sentences, including
the difference between sentences starting with ``before'' and
``after'' and the edge cases that relate decade words like ``twenty''
to non-decade words like ``twenty one.''

  \begin{table}
    \centering
    \begin{tabular}{>{\tiny} l >{\tiny} l >{\tiny} l >{\tiny} l }
      \multicolumn{2}{>{\tiny}l}{$S$(X Y) $\leftarrow$ $A_1$(X, Y) : 1.0000} &      $A_2$(before, comes) : 0.7316 & $A_4$(after, comes) : 0.9990  \\
      \multicolumn{2}{>{\tiny}l}{$A_1$(X Y, U V) $\leftarrow$ $A_2$(X, U), $A_3$(V, Y) : 0.5002} & $A_3$(one, two) : 0.3993 & $A_3$(two, three) : 0.2063 \\
      \multicolumn{2}{>{\tiny}l}{$A_1$(X Y, U V) $\leftarrow$ $A_3$(Y, V), $A_4$(X, U) : 0.3428} & $A_3$(three, four) : 0.1093 & $A_3$(four, five) : 0.0734 \\
      \multicolumn{2}{>{\tiny}l}{$A_1$(X Y, U V) $\leftarrow$ $A_1$(V, Y), $A_4$(X, U) : 0.0796} & $A_3$(five, six) : 0.0502 & $A_3$(six, seven) : 0.0355 \\
      \multicolumn{2}{>{\tiny}l}{$A_1$(X Y, U V) $\leftarrow$ $A_1$(Y, V), $A_2$(X, U) : 0.0712} & $A_3$(eight, nine) : 0.0290 & $A_3$(seven, eight) : 0.0271 \\
      \multicolumn{2}{>{\tiny}l}{$A_1$(X Y, U V) $\leftarrow$ $A_2$(Y, X), $A_3$(V, U) : 0.0021} & $A_2$(fifty, fifty) : 0.0375 & $A_2$(thirty, thirty) : 0.0361 \\
      \multicolumn{2}{>{\tiny}l}{$A_1$(X Y, U V) $\leftarrow$ $A_1$(V, Y), $A_2$(X, U) : 0.0013} & $A_2$(eighty, eighty) : 0.0339 & $A_3$(null, one) : 0.0231 \\
      \multicolumn{2}{>{\tiny}l}{$A_1$(X Y, U V) $\leftarrow$ $A_2$(V, X), $A_3$(Y, U) : 0.0008} & $A_2$(forty, forty) : 0.0332 & $A_2$(twenty, twenty) : 0.0310 \\
      \multicolumn{2}{>{\tiny}l}{$A_1$(X Y, U V) $\leftarrow$ $A_1$(X, U), $A_2$(V, Y) : 0.0008} & $A_2$(seventy, seventy) : 0.0296 & $A_2$(sixty, sixty) : 0.0274 \\
      \multicolumn{2}{>{\tiny}l}{$A_1$(X Y, U V) $\leftarrow$ $A_1$(X, U), $A_2$(Y, V) : 0.0008} & $A_2$(ninety, ninety) : 0.0260 & $A_3$(eighteen, nineteen) : 0.0064 \\
      \multicolumn{2}{>{\tiny}l}{$A_1$(X Y, U V) $\leftarrow$ $A_1$(X, U), $A_4$(Y, V) : 0.0004} & $A_3$(sixteen, seventeen) : 0.0049 & $A_3$(eleven, twelve) : 0.0044 \\
      $A_3$(nine, ten) : 0.0044 & $A_3$(thirteen, fourteen) : 0.0044 & $A_3$(fourteen, fifteen) : 0.0039 & $A_3$(ten, eleven) : 0.0034 \\
      $A_2$(null, fifty) : 0.0043 & $A_3$(eighty, null) : 0.0030 & $A_3$(seventeen, eighteen) : 0.0030 & $A_3$(nine, sixty) : 0.0025 \\
      $A_2$(nine, seventy) : 0.0029 & $A_3$(nine, forty) : 0.0020 & $A_2$(null, thirty) : 0.0022 & $A_2$(nine, ninety) : 0.0022 \\
      $A_3$(twelve, thirteen) : 0.0015 & $A_3$(fifteen, sixteen) : 0.0015 & $A_3$(null, comes) : 0.0010 & $A_2$(twenty, after) : 0.0014 \\
      $A_2$(sixty, before) : 0.0007 & $A_3$(nineteen, comes) : 0.0005 & $A_4$(nine, thirty) : 0.0010 & \\
    \end{tabular}
    \caption{The 4 predicate LPN trained to model sentences in the number domain. Rules with insignificant weights are removed. This LPN generates the completions in Table \ref{tab:results}.}
    \label{tab:grammar}
  \end{table}

\begin{figure}
    \begin{subfigure}[b]{0.45\linewidth}
      \begin{tabular}{>{\footnotesize} l >{\footnotesize} l}
        Question & $K=4$ \\ \hline
        XXXX after twenty comes \underline{\hspace{1cm}}? & twenty one \checkmark \\
        XXXX after forty five comes \underline{\hspace{1cm}}? & forty six \checkmark \\
        XXXX after forty seven comes \underline{\hspace{1cm}}? & forty eight  \checkmark \\
        XXXX after forty nine comes \underline{\hspace{1cm}}? & forty ten $\times$ \\
        XXXX after fifty nine comes \underline{\hspace{1cm}}? & fifty ten $\times$ \\
        XXXX after sixty one comes \underline{\hspace{1cm}}? & sixty two \checkmark \\
        XXXX after sixty three comes \underline{\hspace{1cm}}? & sixty four \checkmark \\
        XXXX after sixty four comes \underline{\hspace{1cm}}? & sixty five \checkmark \\
        XXXX after sixty five comes \underline{\hspace{1cm}}? & sixty six \checkmark \\
        XXXX after sixty nine comes \underline{\hspace{1cm}}? & sixty ten $\times$ \\
        XXXX after seventy three comes \underline{\hspace{1cm}}? & seventy four \checkmark \\
        %% after seventy nine comes \underline{\hspace{1cm}}? & seventy ten $\times$ \\
        %% after ninety five comes \underline{\hspace{1cm}}? & ninety six \checkmark \\
        %% before twenty three comes \underline{\hspace{1cm}}? & twenty two \checkmark \\
        %% before thirty comes \underline{\hspace{1cm}}? & thirty eighty $\times$ \\
        %% before thirty eight comes \underline{\hspace{1cm}}? & thirty seven \checkmark \\
        %% before forty one comes \underline{\hspace{1cm}}? & forty \checkmark \\
        %% before fifty three comes \underline{\hspace{1cm}}? & fifty two \checkmark \\
        %% before sixty eight comes \underline{\hspace{1cm}}? & sixty seven \checkmark \\
        %% before seventy two comes \underline{\hspace{1cm}}? & seventy one \checkmark \\
        %% before seventy three comes \underline{\hspace{1cm}}? & seventy two \checkmark \\
        %% before eighty five comes \underline{\hspace{1cm}}? & eighty four \checkmark \\
        %% before ninety two comes \underline{\hspace{1cm}}? & ninety one \checkmark \\
        %% before ninety three comes \underline{\hspace{1cm}}? & ninety two \checkmark \\
        %% before ninety five comes \underline{\hspace{1cm}}? &ninety four \checkmark \\
      \end{tabular}
      \caption{}
      \label{tab:results}
  \end{subfigure}
  \hfill
  \begin{subfigure}[b]{0.45\linewidth}
    \includegraphics[width=\linewidth]{figures/train_number_net_0006_held_out.pdf}
    \caption{}
    \label{fig:heldoutLL}
  \end{subfigure}
  \caption{(\subref{tab:results}) Viterbi completions of held-out sentences to evaluate an LPN with 4 latent predicates in the number domain. (\subref{fig:heldoutLL}) PUT HELD OUT LL CAPTION HERE}
\end{figure}

To inspect visually the learned grammar, we thresholded rules
according to the expected number of times they were used in parsing
the training dataset. Table~\ref{tab:grammar} shows all rules with
expected count above $1e-6$. This reduces from $2669$ to $52$ the
number of significant rules. On inspection, predicate $A_2$ forms
``before'' sentences, predicate $A_4$ forms ``after'' sentences,
predicate $A_3$ is successorship recursively defined over the decades
and ones, and predicate $A_2$ is a category for the decade words.

Our LPN does not learn to how to transition between the last word in a decade
and the next decade (e.g. ``seventy nine'' to ``eighty''). Instead, it
makes the intuitively reasonable generalization that ``seventy nine''
should be followed by ``seventy ten.'' 

\subsubsection*{Acknowledgments?}

\bibliographystyle{amsplain}
\bibliography{nips_workshop}

\end{document}
